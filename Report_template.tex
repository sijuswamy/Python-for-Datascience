\documentclass{article}
\usepackage[utf8]{inputenc}
\usepackage{graphicx}
\usepackage{hyperref}
\title{Exploratory Data Analysis using iris dataset}
\author{Jithin Mathew}
\date{}

\begin{document}
\maketitle
\tableofcontents
\listoffigures
\newpage
\section{Working with datasets in dataset repositories}

We can play with built-in datasets which are more well organized to understand the basic datascience operations on simple datasets.
\subsection{Introduction}
In this notebook, I have done the Exploratory Data Analysis of the famous Iris dataset and tried to gain useful insights from the data. The features present in the dataset are:
\begin{itemize}
\item Sepal Width
\item Sepal Length
\item Petal Width
\item Petal Length
\end{itemize}
The classes in the species feature and its count in the data set is illustrated in Figure \ref{fig:class_size}.
\begin{figure}[h]
    \centering
    \includegraphics[scale=0.5]{Class_size.pdf}
    \caption{Sampling Distribution of different classes in the Species}
    \label{fig:class_size}
\end{figure}
\end{document}